\documentclass[12pt]{article}
\usepackage[margin=0.5in]{geometry}
\usepackage{amsmath}
\usepackage{amssymb}
\usepackage{graphicx}
\usepackage{silence}
\WarningsOff[latex]
\usepackage{float}
\graphicspath{ {./pics/} }
\linespread{1.2}
\newcommand{\code}{\texttt}

\begin{document}
\subsection*{Exercise 1.}
\noindent (a)(i) We prove that \(\delta\) is well defined by proving its contrapositive statement i.e. 
\begin{equation*}
  \text{if } \exists a \in \Sigma \ [xa] \neq [ya], \text{ then } [x] \neq [y]
\end{equation*}
Now suppose \([xa] \neq [ya]\) for \(a \in \Sigma\), then by the definition of the equivalence classes of \(R_{L}\), \((xa, ya) \notin R_{L}\).
\begin{align*}
  (xa, ya) \notin R_{L} \Leftrightarrow \exists w \in \Sigma^{*}\ (xaw \in L \land yaw \notin L \text{ (or vice versa)})
\end{align*}
Let \(z = aw\), we now have \(xz \in L \land yz \notin L\) (or vice versa).
\begin{align*}
  (xz \in L \land yz \notin L) \Leftrightarrow (x, y) \notin R_{L} \Leftrightarrow [x] \neq [y]
\end{align*}
Hence, we have proven the contrapositive statement.

\noindent (ii) We prove a more general statement i.e. 
\begin{equation*}
  \forall x \in \Sigma^{*}\ \hat{\delta}([\epsilon], x) = [x] 
\end{equation*}
This implies the statement we want to prove since, if \(x \in L\), then by the definition of \(F\), \(\hat{\delta}([\epsilon], x) \in F\); and conversely, if \(\hat{\delta}([\epsilon], x) \in F \Leftrightarrow [x] \in F\), then by the definition of \(F\), \(x \in L\). \\
Now we prove the lemma by an induction on \(|x|\).
\begin{itemize}
  \item \textbf{Basis case} \(x = \epsilon\). \(\delta([\epsilon], \epsilon) = [\epsilon]\) by the definition of the transition function.
  \item \textbf{Step case} Assume that \(\hat{\delta}([\epsilon], x) = [x]\) for \(|x| < k\). We prove for \(w = xa\) where \(x \in \Sigma^{k-1}, a \in \Sigma\).
  \begin{align*}
    \hat{\delta}([\epsilon], xa) &= \delta(\hat{\delta}([\epsilon], x), a) \tag*{(definition of \(\hat{\delta}\))} \\
    &= \delta([x], a) \tag*{(IH)} \\
    &= [xa] \tag*{(definition of \(\delta\))}
  \end{align*}
\end{itemize}
Hence, we have proven the lemma.

\noindent (b).
\end{document}